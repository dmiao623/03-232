\documentclass{article}
\input{preamble}

\begin{document}
\maketitle


\ZS{Task 1}
  \BV{
    \item 
      See Figure \ref{fig:1a}
      \begin{figure}[h]
        \centering
        \includegraphics[width=0.60\textwidth]{fig1a.png}
        \label{fig:1a}
        \caption{Asp-Lys species across protonation states}
      \end{figure}

    \item
      See Figure \ref{fig:1b}
      \begin{figure}[h]
        \centering
        \includegraphics[width=0.40\textwidth]{fig1b.png}
        \label{fig:1b}
        \caption{Titration curve for Asp-Lys}
      \end{figure}

    \item
      C terminus: $\text{pH}=4>\text{p}K_a=2$, so net charge is $-1$ \\
      Asp side chain: $\text{pH}=4=\text{p}K_a=4$, so net charge is $-0.5$ \\
      N terminus chain: $\text{pH}=4<\text{p}K_a=9$, so net charge is $+1$ \\
      Lys side chain: $\text{pH}=4<\text{p}K_a=11$, so net charge is $+1$ \\
      avg. net charge: (+1) + (+1) + (-1) + (-0.5) = \boxed{0.5}

    \item
      C terminus: $\text{pH}=7>\text{p}K_a=2$, so net charge is $-1$ \\
      Asp side chain: $\text{pH}=7>\text{p}K_a=4$, so net charge is $-1$ \\
      N terminus chain: $\text{pH}=7<\text{p}K_a=9$, so net charge is $+1$ \\
      Lys side chain: $\text{pH}=7<\text{p}K_a=11$, so net charge is $+1$ \\
      avg. net charge: (+1) + (+1) + (-1) + (-1) = \boxed{0}

    \item
      C terminus: $\text{pH}=9<\text{p}K_a=2$, so net charge is $-1$ \\
      Asp side chain: $\text{pH}=9<\text{p}K_a=4$, so net charge is $-1$ \\
      N terminus chain: $\text{pH}=9=\text{p}K_a=9$, so net charge is $+0.5$ \\
      Lys side chain: $\text{pH}=9>\text{p}K_a=11$, so net charge is $+1$ \\
      avg. net charge: (-1) + (-0.5) + (+1) + (+1) = \boxed{-0.5}

    \item
      The net $0$ species occurs between p$K_a$ that bracket the neutral form:
      \BE{
        \text{p}I\approx (4+9)/2=\boxed{6.5}
      }
  }

\ZS{Task 2}
  \BV{
    \item
      $0.02\;\mathrm{mol/L}\cdot 0.500\;\mathrm{L} = \boxed{0.010\;\mathrm{mol}}$

    \item
      Use the base, since HEPES starts in the most acidic form $\mathrm{H}_2\mathrm{A}$. 

    \item
      At $\mathrm{pH}=7.1$, the equilibrium that is relevant is the second dissocation between $\mathrm{HA}^-$ and $\mathrm{A}^{2-}$. Using Henderson-Hasselbalch,
      \BE{
        \f{[\mathrm{A}^{2-}]}{[\mathrm{HA}^-]}=10^{7.1-7.5}\approx 0.398.
      }
      Because there are a total of $0.010$ moles, so the number of moles of $\mathrm{A}^{2-}$ is $0.00285$ and the number of moles of $\mathrm{HA}^-$ is $0.00715$. It takes two moles of $\mathrm{OH}^-$ to make the former, and one mole for the latter. Then, we can take a weighted average:
      \BE{
        \mathrm{moles\;of\;OH}^-=1\cdot 0.00715 + 2\cdot 0.00285 = 0.01285\;\mathrm{mol}.
      }
      Because $[\text{KOH}]=1\mathrm{ M}$, this is equivalent to $\boxed{12.85\;\mathrm{mL}}$ of KOH.
  } 

\ZS{Task 3}
  \BV{
    \item
      At $\mathrm{pH}=7.1$, the buffer has $7.15\;\mathrm{mmol}$ of $\mathrm{HA}^-$ and $2.85\;\mathrm{mmol}$ of $\mathrm{A}^{2-}$. Adding base converts the protonated form to the unprotonated form, so there will be $7.10\;\mathrm{mmol}$ and $2.90\;\mathrm{mmol}$ of each, respectively. Appyling Henderson-Hasselbalch gives
      \BE{
        \mathrm{pH}=7.5+\log(2.90/7.10)\approx\boxed{7.11}
      }

    \item
      The $\mathrm{pH}$ only changed by $0.01$, a relatively insignificant change.

    \item
      The KOH dissocatiates completely, so $[\mathrm{OH}^-]=5\cdot 10^5\;\mathrm{mol}/0.50\;\mathrm{L}=10\cdot 10^-4\;\mathrm{M}$. Taking the negative log gives a $\mathrm{pH}$ of $\boxed{10.00}$.
  }
  

\ZS{Task 4}
  \BV{
    \item
      We have
      \BE{
        \varepsilon_\text{protein}=3\varepsilon(\text{Trp})+3\varepsilon(\text{Tyr})=15150+4470=19620\;\mathrm{M}^{-1}\mathrm{cm}^{-1}.
      }
      Then,
      \BE{
        c=\f{A}{\epsilon\ell}=\boxed{5.61\cdot 10^{-5}\;\mathrm{M}}.
      }

    \item
      There are $cV=(5.61\cdot 10^-5)(1.0\cdot 10^-3)=5.61\cdot 10^{-8}$ moles of the protein. Then, the mass is $(5.61\cdot10^-8)(20000)=\boxed{1.12\cdot 10^-3\;\mathrm{g}}$.
  }

  
\end{document}

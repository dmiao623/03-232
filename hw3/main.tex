ldocumentclass{article}
\input{preamble}

\begin{document}
\maketitle

\ZS{Task 1}
  \BR{
    \item 
      Residue 7 (second from last) / Residue 8 (last)

    \item
      $3.1$ \AA

    \item
      $\phi=-65.0^\circ, \psi=-40.0^\circ$
  }

\ZS{Task 2}
  \BR{
    \item
      $1$ H-bonds per residue.

    \item
      $2.8$ \AA

    \item
      In an $\alpha$-helix, the H-bonds are within the same polypeptide chain: the carbonyl of residue $i$ H-bonds to the amide N–H of residue $i+4$. In a $\beta$-sheet, the hydrogen bonds are between neighboring strands.

    \item
      $\phi=-151.8^\circ, \psi=146.0^\circ$
  }

\ZS{Task 3}
  \BR{
    \item
      $139$ amino acids

    \item
      Proline (P).

    \item
      The protein contains a ligand on the inside. This can be determined by highlighting the \texttt{RTL} in the sequence and coloring by atom.

    \item
      Antiparallel.

    \item
      Yes. One face contains many compounds with R groups containing many oxygens and nitrogens, which enables the formation of H-bonds, while the other face consists of mostly aliphatic and aromatic compounds. This is likely because when the protein folds into its 3D conformation, the lowest energy state is achieved when polar R groups can H-bond with water and non-polar R groups can interact with one another through van der Waals forces.

    \item
      The ligand is an alkane with an phenyl on one end and an alcohol group on the other.

    \item
      The ligand interacts with the protein primarily through van der Waals forces, because it is a long non-polar molecules with no permanent dipoles. From Part (v), the inside of the protein mostly consists of non-polar R groups, which encourages bonding through vdW forces.
      

    \item
      The ligand acts as an H-bond donor to Q108 and an H-bond acceptor from K40.
  }
  
\end{document}
